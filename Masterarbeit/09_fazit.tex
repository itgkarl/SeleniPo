\chapter{Fazit}
\label{sec:fazit}
Das manuelle Testen von Software ist ein Bereich der Softwareentwicklung, der in der Literatur mittlerweile gut und umfassend erläutert wird. Anders verhält es sich mit der Testautomatisierung. Dieser Bereich des Testens ist zwar schon lange bekannt, wird in der Literatur jedoch immer noch nicht umfassend genug behandelt. Mit dieser Arbeit wurde ein Werk geschaffen, welches sich mit einer Sparte der Softwareentwicklung befasst, die noch Entwicklungspotential in sich birgt. 
Die Arbeit behandelte zunächst die Grundlagen des Testens im Allgemeinen und hat herausgearbeitet, dass Testen eine sinnvolle Möglichkeit ist, um die Qualität von Softwareprodukten zu verbessern. 
Im weiteren Verlauf rückte die Automatisierung von Testfällen in den näheren Fokus. Die verschiedenen Bereiche und Möglichkeiten der Testautomatisierung wurden aufgezeigt, sowie die Vor- und Nachteile herausgearbeitet.
Die Testautomatisierung zeichnete sich dabei als Möglichkeit ab, die Erfolge, die mit Hilfe von Softwaretests erzielt werden, weiter zu verbessern und gleichzeitig die dabei eingesetzten Mittel zu reduzieren. Als ein Problem der Testautomatisierung zeigte sich jedoch der oft erhöhte initiale Mehraufwand bei der Erstellung der Tests. Dieses Problem wurde auch als mögliches Verbesserungspotential in der Testautomatisierung erkannt.\\
Um die gewonnenen Erkenntnisse anhand eines Beispiels zu konkretisieren, wurde das Tool Selenium vorgestellt.
Auch für dieses Tool hat sich gezeigt, dass der initiale Mehraufwand bei der Testfallerstellung, verglichen mit der manuellen Durchführung der Testfälle, erhöht ist, vor allem dann, wenn Selenium wie vorgeschlagen in Verbindung mit dem Page Object Pattern verwendet wird.\\
Für den Einsatz von Selenium in Verbindung mit dem Page Object Pattern wurde daher eine Softwarelösung entwickelt und vorgestellt, welche den Aufwand bei der Erstellung von automatischen Tests reduziert.
Die Softwarelösung in Form eines Page Object Generators vereinfacht dazu die Erstellung der Page Objects die beim Einsatz des Page Object Patterns benötigt werden. Page Objects müssen nicht mehr händisch programmiert, sondern können teilautomatisiert aus dem Quelltext einer Webseite generiert werden. Damit trägt der Page Object Generator dazu bei, den initialen Mehraufwand bei der Erstellung des Testfälle zu reduzieren.

\section{Ausblick}
\label{ausblick}
Der Page Object Generator hat einen Stand erreicht, der sich für den produktiven Einsatz eignet. Praxistests zeigten jedoch, dass Generator sowie der zugehörige Testharness, durchaus noch Verbesserungspotenzial bieten.
Ein Maximum an Zeitersparnis lässt sich mit dem Page Object Generator erreichen, wenn der Konfigurationsaufwand von Testprojekt und Generator möglichst gering gehalten wird.
Ein sinnvoller Ansatzpunkt, um das Projekt SeleniPo weiter voranzutreiben, wäre daher, die Entwicklung des als Testharness bezeichneten Testprojekts zu verbessern. Bei der Implementierung des Testharness, wie in Kapitel \ref{sec:selenipotestharness} beschrieben, handelt es sich um einen Prototypen, der zwar als Grundlage für den produktiven Einsatz verwendet werden kann, jedoch noch unausgereift ist. Durch einen perfekt auf den Generator abgestimmten Testharness könnte die Hemmschwelle für die Benutzung des Tools gesenkt werden, was wiederum die Akzeptanz beim Anwender steigern würde.\\
Darüber hinaus lässt sich mit einem gut vorbereiteten Testharness die Verwendung von Best Practice Ansätzen über das Page Object Pattern hinaus unterstützen.\\
Ein weitere Möglichkeit, um den der Page Object Generator zu verbessern, wäre, verschiedene Templates für die Generierung der Page Objects anzubieten.
Die derzeit im Page Object Generator angebotenen Templates und damit die generierten Page Object Klassen haben sich vom Selenium vorgeschlagenen Standard für Page Objects entfernt. Selenium schlägt die Verwendung von Page Objects mit annotierten WebElements als Variablen vor. Die mit Locatoren annotierten WebElemente werden bei der Instanziierung über eine PageFactory-Klasse aufgelöst und befüllt.
Im Gegensatz dazu erzeugen die im Page Object Generator derzeit hinterlegten Templates Page Object-Klassen, die an Stelle von WebElements die Klasse Control verwenden und abhängig von der der Klasse ByFactory sind (siehe Kapitel \ref{sec:selenipotestharness}). Um die von Selenium angebotene PageFactory zu unterstützen, müssen neue Templates geschaffen werden, welche die von der PageFactory-Klasse benötigten Konventionen erfüllen.
Templates, die den von Selenium vorgeschlagenen Standard erfüllen, hätten den Vorteil, dass sie den Einstieg für Entwickler, die bereits Erfahrungen mit Selenium und dem WebDriver gesammelt haben, erleichtern würden.\\
Unabhängig davon, für welchen Weg sich bei der Weiterentwicklung des Generators entschieden wird, ist es für die Landeshauptstadt München vor allem wichtig, den Generator nun über den Pilotbetrieb hinaus in den produktiven Einsatz auszurollen.












